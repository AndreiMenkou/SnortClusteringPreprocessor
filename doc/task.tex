Положим X={x_1, x_2, ..., x_n} - множество узлов в компьтерной сети.
Пусть каждый узел x_i \in X в момент времени t \in T характеризуется состоянием S(x_i, t)
Состоянием сети S(X) будем называть множество состояний её узлов S(X, t) = {S(x_i, t) | x_i \in X}
Будем считать, что узлы взаимодействуют между собой посредством передачи сообщений, используя сетевой протокол. Тогда положим m(x_i, x_j) - управляющая информация от объекта x_i к x_j. Назовём переходом изменение состояния узла в результате взаимодействия с участием этого узла. 

Введём множества состояний A и N (от. Attack и Normal соответственно).
A - множество состояний узлов, каждое из которых представляет состояние узла после произведения над ним какой-либо компьютерной атаки
N - множество нормальных состояний узлов

Состояние сети будем называть опасным, если состояние хотя бы одного узла в этой сети принадлежит множеству A.
Таким образом для обнаружения атак в такой сети достаточно наблюдать за состояниями узлов этой сети, а точнее за изменением состояний этих узлов. 
В рамках данной работы будем предполагать, что состояния узлов изменяются только в результате взаимодействия узлов между собой (ввиду того, что предметом исследования являются атаки на компьютерные сети).
Зафиксируем узел сети x \in X. 
Пусть в момент времени t произошло взаимодействие узлов x и y в сети, в результате которого на узел x поступила управляющая информация I. В ответ на это узел x выполняет действия, которые в дальнейшем будем называть реакцией узла и обозначать R = f(I), где f - функция реагирования с областью определения D(f) = {множество всех возможных входов}. По сути эта функция реализована в виде механизма работы конкретного узла x сети и вообще говоря может отличаться для разных узлов. Она и реализует смену состояний узла x \in X.

Задачу обнаружения атак в компьютерной сети можно теперь записать в следующем виде:
F(X) -> min
g(X) 
