%conclusion
\section*{ЗАКЛЮЧЕНИЕ}
\addcontentsline{toc}{section}{ЗАКЛЮЧЕНИЕ}

За время прохождения преддипломной практики выполнено:

\begin{itemize}

\item изучен предмет компьютерной атаки, их основные типы и угрозы, которые они могут представлять

\item рассмотрены и проанализированы существующие на данный момент методы обнаружения компьютерных
атак, их реализации в современных системах обнаружения атак

\item выявлены достоинства и недостатки существующих методов

\item найдены и получены актуальные тестовые данные о поведении в компьютерной сети 
за 2012 год \cite{bib:iscx2012}

\item предложена модификация существующих методов обнаружения аномалий (снято ограничение
на необходимость иметь на входе алгоритма кластеризованные по типам данные)

\item рассмотрены основные алгоритмы кластерного анализа

\end{itemize}

В итоге можно сделать вывод, что была проделана работа по изучению теоретической основы, на
которой базируется любая система обнаружения атак, изучены основные подходы к решению проблемы
обнаружения атак и предложен вариант модификации одного из методов обнаружения аномалий 
с использованием инструмента кластерного анализа.
Дальнейшая работа предполагает реализацию предложенного алгоритма и внедрение его как 
составной части комплексной системы обнаружения атак Snort.

\newpage

%conclusion
\addcontentsline{toc}{section}{СПИСОК ИСПОЛЬЗОВАННЫХ ИСТОЧНИКОВ}
\renewcommand{\refname}{СПИСОК ИСПОЛЬЗОВАННЫХ ИСТОЧНИКОВ}

\begin{thebibliography}{99}
    
    \bibitem{bib:gostAttack}
    Защита информации. Объект информатизации. Факторы, воздействующие
    на информацию. Общие положения: ГОСТ Р 51275-2006. --- Введ. --- М.~
    Стандартинформ, 2007. --- 7 с.
    
    \bibitem{bib:phys_attack_types}
    Губенков, А.~А. Информационная безопасность / А.~А. Губенков, В.~Б. Байбурин. ---
    М.: Новый издательский дом, 2005. --- 128 с.
    
    \bibitem{bib:ids}
    Bace, R. An Introduction to Intrusion Detection \& Assessment: For System and Network 
    Security Management / R.~Bace // ICSA White Paper. --- 1998. --- 38 p.
    
    \bibitem{}
	Northcutt, S. Network Intrusion Detection. / S.~Northcutt, J.~Novak. --- 
	New Riders Publishing, 2002. --- 346 p.
	
	\bibitem{bib:bro}
	Bro Intrusion Detection System [Electronic resource] --- Mode of access: 
	http://www.bro-ids.org/. --- Date of access: 15.01.2013.
	
	\bibitem{bib:ossec}
	OSSEC --- an Open Source Host-based Intrusion Detection System [Electronic resource] 
	--- Mode of access: http://www.ossec.net/. --- Date of access: 15.01.2013.
	
	\bibitem{bib:netstat}
	Vigna, G. NetSTAT: A Network-based Intrusion Detection System. / G.~Vigna, 
	R.~A. Kemmerer --- Journal of Computer Security, 1999. --- 79 p.
	
	\bibitem{bib:prelude}
	Prelude SIEM [Electronic resource]	--- Mode of access: 
	https://www.prelude-ids.org/. --- Date of access: 18.01.2013.
	
	\bibitem{bib:snort}
	Snort network intrusion prevention and detection system [Electronic resource] ---
	Mode of access: http://www.snort.org/. --- Date of access: 20.01.2013.
	
	\bibitem{bib:iscx2012}
	UNB ISCX Intrusion Detection Evaluation DataSet --- Information Security Center of eXcellence
	[Electronic resource] --- Mode of access: http://www.iscx.ca/dataset.~---	Date of access: 25.11.2012. 
	
\end{thebibliography}