%titlepage
\titlepage
\begin{center}
	\begin{large}
		\textbf{МИНИСТЕРСТВО~ОБРАЗОВАНИЯ~РЕСПУБЛИКИ~БЕЛАРУСЬ}
		
		\smallskip
		\textbf{БЕЛОРУССКИЙ~ГОСУДАРСТВЕННЫЙ~УНИВЕРСИТЕТ}
		
		\smallskip
		\textbf{Факультет~прикладной~математики и информатики}
		
		\smallskip
		Кафедра~информационных~систем~управления
	\end{large}
\end{center}

\vfill

\begin{center}
	\large {МЕНЬКОВ АНДРЕЙ АЛЕКСАНДРОВИЧ}
	
	\bigskip	
	{\Large \textbf{РАЗРАБОТКА МАТЕМАТИЧЕСКОГО И ПРОГРАММНОГО ОБЕСПЕЧЕНИЯ ДЛЯ РЕШЕНИЯ ЗАДАЧИ ОБНАРУЖЕНИЯ АТАК НА КОМПЬЮТЕРНЫЕ СИСТЕМЫ}}
	
	\bigskip
	Дипломная работа
	
	студента 5 курса 2 группы
\end{center}


\vfill
\begin{flushright}
	\begin{minipage}{7cm}
		\textbf{Руководитель:}
		
		\textit{Образцов Владимир Алексеевич}
		
		доцент кафедры ИСУ,
		
		кандидат физико-математических наук 
	\end{minipage}
\end{flushright}

\vfill

\begin{center}
	МИНСК
	
	БГУ
	
	2013
\end{center}

\newpage

%abstract
\newcommand{\pagescount}{24}
\section*{АННОТАЦИЯ}

\textit{Меньков А. А}. Разработка математического и программного обеспечения для решения задачи обнаружения атак на компьютерные системы:
Дипломная работа~/ Минск: 
БГУ, 2013. --- \pagescount~с.

\medskip


\section*{АНАТАЦЫЯ}

\textit{Менькоу А. А.} Распрацоўка матэматычнага і праграмнага забеспячэння для вырашэння задачы выяўлення нападаў на кампутарныя сістэмы: Дыпломная работа~/ Мінск: 
БДУ, 2013. --- \pagescount c.

\medskip


\section*{ANNOTATION}

\textit{Menkou A. A.} Development of mathematical and software solutions for the problem of detection of attacks on computer systems:
Dyploma~/ Minsk: BSU, 2013 --- \pagescount p.

\medskip

\newpage

%summary
\section*{РЕФЕРАТ}

Дипломная работа, \pagescount с.
%** табл., ? источников. 

\medskip
\textbf{Ключевые слова:} КОМПЬЮТЕРНАЯ СЕТЬ, АТАКА, ОБНАРУЖЕНИЕ АТАК.

\medskip
\textbf{Объект исследования} --- атаки на компьютерные сети.

\textbf{Цель работы} --- разработать обеспечение для обнаружения атак на компьютерные сети.

\textbf{Методы исследования} --- методы прикладной математики и информатики, 
кластерный анализ.

\textbf{Результат исследования} --- построение модели обнаружения атак в компьютерной сети и графического интерфейса пользователя для оценки эффективности построенной модели.

\textbf{Областью применения} являются компьютерные сети с повышенным контролем безопасности информации, передаваемой по сети.

\newpage

%contents
\renewcommand{\contentsname}{СОДЕРЖАНИЕ}
\tableofcontents

% \newpage
% \listoffigures
\newpage

%introduction
\section*{ВВЕДЕНИЕ}
\addcontentsline{toc}{section}{ВВЕДЕНИЕ}

Компьютерные сети за несколько последних десятилетий из чисто технического решения превратились в глобальное явление, развитие которого оказывает влияние на большинство сфер экономической деятельности. Одним из первых количественную оценку значимости сетей дал Роберт Меткалф, участвовавший в создании Ethernet: по его оценке <<значимость>> сети во всех смыслах пропорциональна квадрату числа узлов в ней. То есть, зависимость от нормальной работы сетей растёт быстрее, чем сами сети. Обеспечение работоспособности сети и функционирующих в ней информационных систем зависит не только от надёжности аппаратуры, но и, зачастую, от способности сети противостоять целенаправленным воздействиям, которые направлены на нарушение её работы.

Создание информационных систем, гарантированно устойчивых к вредоносным воздействиям и компьютерным атакам, сопряжено с существенными затратами как времени, так и материальных ресурсов. Кроме того, существует известная обратная зависимость между удобством пользования системой и её защищённостью: чем совершеннее системы защиты, тем сложнее пользоваться основным функционалом информационной системы. В 80-е годы XX века, в рамках оборонных проектов США, предпринимались попытки создания распределенных информационных систем специального назначения (MMS – Military Messaging System), для которых формально доказывалась выполнимость основной теоремы безопасности – невыведение системы из безопасного состояния для любой последовательности действий взаимодействующих объектов. В этих системах использовалось специализированное программное обеспечение на всех уровнях, включая системный. Однако, на сегодняшний день подобные системы не получили развития, и для организации информационных систем используются операционные системы общего назначения, такие как ОС семейства Microsoft Windows, GNU/Linux, *BSD и различные клоны SysV UNIX (Solaris, HP-UX, etc).

Методы обнаружения атак в современных системах обнаружения атак (далее - СОА) недостаточно проработаны в части формальной модели атаки, и, следовательно, для них достаточно сложно строго оценить такие свойства как вычислительная сложность, корректность, завершимость. Принято разделять методы обнаружения атак на методы обнаружения аномалий и методы обнаружения злоупотреблений. Ко второму типу методов относятся большинство современных коммерческих систем (Cisco IPS, ISS RealSecure, NFR) --- они используют сигнатурные (экспертные) методы обнаружения. Для таких систем основной проблемой является низкая, близкая к нулю, эффективность обнаружения неизвестных атак (адаптивность). Низкая адаптивность до сих пор остаётся проблемой, хотя такие достоинства как низкая вычислительная сложность и малая стоимость развёртывания определяют доминирование таких систем в данной области.

В данной работе реализуется попытка создать модель обнаружения аномалий в компьютерной системе на основе <<сырых>> сетевых данных, собранных c компьютеров в сети с реализацией как компьютерных атак, так и нормального поведения в сети.
