%titlepage
\titlepage
\begin{center}
	\begin{large}
		\textbf{МИНИСТЕРСТВО~ОБРАЗОВАНИЯ~РЕСПУБЛИКИ~БЕЛАРУСЬ}
		
		\smallskip
		\textbf{БЕЛОРУССКИЙ~ГОСУДАРСТВЕННЫЙ~УНИВЕРСИТЕТ}
		
		\smallskip
		\textbf{Факультет~прикладной~математики и информатики}
		
		\smallskip
		Кафедра~информационных~систем~управления
	\end{large}
\end{center}

\vfill

\begin{center}
	\large {ТУМАШИК ИГОРЬ АЛЕКСАНДРОВИЧ}
	
	\bigskip	
	{\Large \textbf{ПОЛУТОНОВЫЕ МАТРИЧНЫЕ ШТРИХКОДЫ}}
	
	\bigskip
	Дипломная работа
	
	студента 5 курса 2 группы
\end{center}


\vfill
\begin{flushright}
	\begin{minipage}{7cm}
		\textbf{Руководитель:}
		
		\textit{Абламейко Сергей Владимирович}
		
		академик, профессор кафедры ИСУ,
		
		доктор технических наук 
	\end{minipage}
\end{flushright}

\vfill

\begin{center}
	МИНСК
	
	БГУ
	
	2013
\end{center}

\newpage

%abstract
\newcommand{\pagescount}{10}
\section*{АННОТАЦИЯ}

\textit{Тумашик И. А}. Полутоновые матричные штрихкоды:
Дипломная работа~/ Минск: БГУ, \\ 2013.~--- \pagescount~с.

\medskip


\section*{АНАТАЦЫЯ}

\textit{Tумашык І. А.} Паўтонавыя матрычныя штрыхкоды:
Дыпломная работа~/ Мінск: БДУ, \\ 2013.~--- \pagescount~c.

\medskip


\section*{ANNOTATION}

\textit{Tumashyk I. A.} Grayscale matrix barcodes:
Dyploma~/ Minsk: BSU, 2013~--- \pagescount~p.

\medskip

\newpage

%summary
\section*{РЕФЕРАТ}

Дипломная работа, \pagescount\ с., ? рис., 
%** табл.,
? источников. 

\medskip
\textbf{Ключевые слова:} ШТРИХКОД, МАТРИЧНЫЙ ШТРИХКОД, 
РАСПОЗНАВАНИЕ ОБРАЗОВ, ОБРАБОТКА ИЗОБРАЖЕНИЙ.

\medskip
\textbf{Объект исследования} --- матричные штрихкоды, полутоновые штрихкоды.

\textbf{Цель работы} --- разработать спецификацию полутонового матричного 
штрихкода, предложить реализацию.

\textbf{Методы исследования} --- методы прикладной математики и информатики, 
технология программирования.

\textbf{Результат исследования} --- разработан полутоновый матричный 
штрихкод, представлена реализация. 

\textbf{Областью применения} являются системы использующие автоматическую
идентификацию объектов находящихся в прямой видимости.

\newpage

%contents
\renewcommand{\contentsname}{СОДЕРЖАНИЕ}
\tableofcontents

\newpage
\listoffigures
\listoftables
\newpage



%introduction
\section*{ВВЕДЕНИЕ}
\addcontentsline{toc}{section}{ВВЕДЕНИЕ}

Уже давно вычислительная техника используется не только в прямом 
взаимодействии с другими цифровыми устройствами, но и с предметами,
имеющими аналоговую природу. В частности, в этой работе пойдёт речь о
применении ЭВМ для считывания графических данных, 
непосредственно предназначенных 
для этого, посредством устройств ввода изображений. Важным аспектом
является как раз то, что рассматриваемые объекты специально спроектированы
для распознавания их цифровыми устройствами. Среди них особое место 
занимают \textit{штрихкоды}\footnote{В настоящее время очень часто
это слово пишут через девиз (см., например, <<Википедию>>). Однако,
<<Русский орфографический словарь: около 180 000 слов>> под редакцией
Лопатина В.~В. \cite{bib:russkijLopatin} настаивает на слитном написании~--- 
\textit{штрихкод}, в таком виде это слово и будет использовано в работе.}. 

Одной из важнейших характеристик любого штрихкода является площадь
занимаемая на рабочей поверхности. В то же время понятно, что невозможно
бесконечно уменьшать размеры элементов кода (по условиям печати и 
качества распознавания). Вывод напрашивается сам собой: следует
увеличить информативность наименьшего элемента кода, чтобы добиться
уменьшения размера всего штрихкода. Логично использовать для этого
градации яркости~--- от чёрного до белого (тон, всё-таки, очень зависит
от освещённости). Будем такие коды называть \textit{полутоновыми}. 
Ещё более реальной делают эту идею всё возрастающее
мощности цифровых камер различных устройств. Вокруг этой простой 
задумки и построена данная работа. Кроме приведения спецификации 
полутонового кода, рассматриваются существующие разработки в этой 
области, проводиться сравнение с существующими подходами (автору не 
известны другие штрихкоды построенные на этой идеи), приводиться
реализация полутонового штрихкода. 

